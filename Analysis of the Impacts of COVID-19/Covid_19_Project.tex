% Options for packages loaded elsewhere
\PassOptionsToPackage{unicode}{hyperref}
\PassOptionsToPackage{hyphens}{url}
%
\documentclass[
]{article}
\usepackage{amsmath,amssymb}
\usepackage{iftex}
\ifPDFTeX
  \usepackage[T1]{fontenc}
  \usepackage[utf8]{inputenc}
  \usepackage{textcomp} % provide euro and other symbols
\else % if luatex or xetex
  \usepackage{unicode-math} % this also loads fontspec
  \defaultfontfeatures{Scale=MatchLowercase}
  \defaultfontfeatures[\rmfamily]{Ligatures=TeX,Scale=1}
\fi
\usepackage{lmodern}
\ifPDFTeX\else
  % xetex/luatex font selection
\fi
% Use upquote if available, for straight quotes in verbatim environments
\IfFileExists{upquote.sty}{\usepackage{upquote}}{}
\IfFileExists{microtype.sty}{% use microtype if available
  \usepackage[]{microtype}
  \UseMicrotypeSet[protrusion]{basicmath} % disable protrusion for tt fonts
}{}
\makeatletter
\@ifundefined{KOMAClassName}{% if non-KOMA class
  \IfFileExists{parskip.sty}{%
    \usepackage{parskip}
  }{% else
    \setlength{\parindent}{0pt}
    \setlength{\parskip}{6pt plus 2pt minus 1pt}}
}{% if KOMA class
  \KOMAoptions{parskip=half}}
\makeatother
\usepackage{xcolor}
\usepackage[margin=1in]{geometry}
\usepackage{color}
\usepackage{fancyvrb}
\newcommand{\VerbBar}{|}
\newcommand{\VERB}{\Verb[commandchars=\\\{\}]}
\DefineVerbatimEnvironment{Highlighting}{Verbatim}{commandchars=\\\{\}}
% Add ',fontsize=\small' for more characters per line
\usepackage{framed}
\definecolor{shadecolor}{RGB}{248,248,248}
\newenvironment{Shaded}{\begin{snugshade}}{\end{snugshade}}
\newcommand{\AlertTok}[1]{\textcolor[rgb]{0.94,0.16,0.16}{#1}}
\newcommand{\AnnotationTok}[1]{\textcolor[rgb]{0.56,0.35,0.01}{\textbf{\textit{#1}}}}
\newcommand{\AttributeTok}[1]{\textcolor[rgb]{0.13,0.29,0.53}{#1}}
\newcommand{\BaseNTok}[1]{\textcolor[rgb]{0.00,0.00,0.81}{#1}}
\newcommand{\BuiltInTok}[1]{#1}
\newcommand{\CharTok}[1]{\textcolor[rgb]{0.31,0.60,0.02}{#1}}
\newcommand{\CommentTok}[1]{\textcolor[rgb]{0.56,0.35,0.01}{\textit{#1}}}
\newcommand{\CommentVarTok}[1]{\textcolor[rgb]{0.56,0.35,0.01}{\textbf{\textit{#1}}}}
\newcommand{\ConstantTok}[1]{\textcolor[rgb]{0.56,0.35,0.01}{#1}}
\newcommand{\ControlFlowTok}[1]{\textcolor[rgb]{0.13,0.29,0.53}{\textbf{#1}}}
\newcommand{\DataTypeTok}[1]{\textcolor[rgb]{0.13,0.29,0.53}{#1}}
\newcommand{\DecValTok}[1]{\textcolor[rgb]{0.00,0.00,0.81}{#1}}
\newcommand{\DocumentationTok}[1]{\textcolor[rgb]{0.56,0.35,0.01}{\textbf{\textit{#1}}}}
\newcommand{\ErrorTok}[1]{\textcolor[rgb]{0.64,0.00,0.00}{\textbf{#1}}}
\newcommand{\ExtensionTok}[1]{#1}
\newcommand{\FloatTok}[1]{\textcolor[rgb]{0.00,0.00,0.81}{#1}}
\newcommand{\FunctionTok}[1]{\textcolor[rgb]{0.13,0.29,0.53}{\textbf{#1}}}
\newcommand{\ImportTok}[1]{#1}
\newcommand{\InformationTok}[1]{\textcolor[rgb]{0.56,0.35,0.01}{\textbf{\textit{#1}}}}
\newcommand{\KeywordTok}[1]{\textcolor[rgb]{0.13,0.29,0.53}{\textbf{#1}}}
\newcommand{\NormalTok}[1]{#1}
\newcommand{\OperatorTok}[1]{\textcolor[rgb]{0.81,0.36,0.00}{\textbf{#1}}}
\newcommand{\OtherTok}[1]{\textcolor[rgb]{0.56,0.35,0.01}{#1}}
\newcommand{\PreprocessorTok}[1]{\textcolor[rgb]{0.56,0.35,0.01}{\textit{#1}}}
\newcommand{\RegionMarkerTok}[1]{#1}
\newcommand{\SpecialCharTok}[1]{\textcolor[rgb]{0.81,0.36,0.00}{\textbf{#1}}}
\newcommand{\SpecialStringTok}[1]{\textcolor[rgb]{0.31,0.60,0.02}{#1}}
\newcommand{\StringTok}[1]{\textcolor[rgb]{0.31,0.60,0.02}{#1}}
\newcommand{\VariableTok}[1]{\textcolor[rgb]{0.00,0.00,0.00}{#1}}
\newcommand{\VerbatimStringTok}[1]{\textcolor[rgb]{0.31,0.60,0.02}{#1}}
\newcommand{\WarningTok}[1]{\textcolor[rgb]{0.56,0.35,0.01}{\textbf{\textit{#1}}}}
\usepackage{graphicx}
\makeatletter
\def\maxwidth{\ifdim\Gin@nat@width>\linewidth\linewidth\else\Gin@nat@width\fi}
\def\maxheight{\ifdim\Gin@nat@height>\textheight\textheight\else\Gin@nat@height\fi}
\makeatother
% Scale images if necessary, so that they will not overflow the page
% margins by default, and it is still possible to overwrite the defaults
% using explicit options in \includegraphics[width, height, ...]{}
\setkeys{Gin}{width=\maxwidth,height=\maxheight,keepaspectratio}
% Set default figure placement to htbp
\makeatletter
\def\fps@figure{htbp}
\makeatother
\setlength{\emergencystretch}{3em} % prevent overfull lines
\providecommand{\tightlist}{%
  \setlength{\itemsep}{0pt}\setlength{\parskip}{0pt}}
\setcounter{secnumdepth}{-\maxdimen} % remove section numbering
\ifLuaTeX
  \usepackage{selnolig}  % disable illegal ligatures
\fi
\usepackage{bookmark}
\IfFileExists{xurl.sty}{\usepackage{xurl}}{} % add URL line breaks if available
\urlstyle{same}
\hypersetup{
  hidelinks,
  pdfcreator={LaTeX via pandoc}}

\author{}
\date{\vspace{-2.5em}}

\begin{document}

Analysis of the Impacts of COVID-19 on Global Populations

\textbf{Author}: Willian Botelho

Date: 05/12/2024

\subsubsection{Introduction}\label{introduction}

This academic project conducts an analysis of COVID-19 data, which
includes daily records of confirmed cases and virus-related deaths. The
data were collected and consolidated from various global sources and are
essential for understanding the impacts of the pandemic, especially in
countries with populations larger than that of Brazil.

\subsubsection{Data Description}\label{data-description}

The datasets used in this study include several key variables, such as:

\begin{itemize}
\tightlist
\item
  \textbf{FIPS} (Federal Information Processing Standards): A code used
  in the United States to uniquely identify counties.
\item
  \textbf{Admin2}: The name of the county, applicable only in the United
  States.
\item
  \textbf{Province\_State}: The name of the province or state.
\item
  \textbf{Country\_Region}: The name of the country or region, as
  officially designated by the U.S. Department of State.
\item
  \textbf{Last Update}: The date and time of the last data update, in
  MM/DD/YYYY HH:mm:ss UTC format.
\item
  \textbf{Lat and Long\_}: Geographic coordinates, with representative
  centroids for each location.
\item
  \textbf{Confirmed}: Number of confirmed and probable cases.
\item
  \textbf{Deaths}: Number of confirmed and probable deaths.
\end{itemize}

\subsubsection{Study Objectives}\label{study-objectives}

The primary objective of this study is to assess the impacts of COVID-19
on countries with populations larger than Brazil, using consolidated
data that includes key variables for effective analysis. This work is
part of the Master's in Data Science program at the University of
Colorado Boulder and aims to deepen understanding of the pandemic,
without intending to influence public health policies.

\subsubsection{Data Sources}\label{data-sources}

The data for this study were extracted from the following GitHub
repositories:

\begin{itemize}
\item
  Daily data on confirmed cases:
  \href{https://github.com/CSSEGISandData/COVID-19/blob/master/csse_covid_19_data/csse_covid_19_time_series/time_series_covid19_confirmed_global.csv}{COVID-19
  Time Series Data - Confirmed Cases}
\item
  Daily data on deaths:
  \href{https://github.com/CSSEGISandData/COVID-19/blob/master/csse_covid_19_data/csse_covid_19_time_series/time_series_covid19_deaths_global.csv}{COVID-19
  Time Series Data - Deaths}
\item
  Country population reference:
  \href{https://github.com/CSSEGISandData/COVID-19/blob/master/csse_covid_19_data/UID_ISO_FIPS_LookUp_Table.csv}{COVID-19
  UID ISO FIPS LookUp Table}
\end{itemize}

This study seeks not only to understand but also to systematically
document the patterns of dissemination and impact of the virus,
contributing to future research and interventions in global public
health.

\begin{Shaded}
\begin{Highlighting}[]
\NormalTok{general\_url }\OtherTok{=} \StringTok{"https://github.com/CSSEGISandData/COVID{-}19/tree/master/csse\_covid\_19\_data/csse\_covid\_19\_time\_series/"}
\NormalTok{confirmed\_global }\OtherTok{=} \StringTok{"time\_series\_covid19\_confirmed\_global.csv"}
\NormalTok{deaths\_global   }\OtherTok{=} \StringTok{"time\_series\_covid19\_deaths\_global.csv"}
\NormalTok{UID\_FIPS\_table }\OtherTok{=} \StringTok{"https://github.com/CSSEGISandData/COVID{-}19/blob/master/csse\_covid\_19\_data/UID\_ISO\_FIPS\_LookUp\_Table.csv"}
\end{Highlighting}
\end{Shaded}


\end{document}
